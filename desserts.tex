\chapter{Desery}

\section{Crème anglaise}

\begin{ingreds}
\item{sugar, 0.125}
\item{yolks, 2}
\item{milk, 0.85}
\opitem{vanilla extract, 0.5 Tb}
\opitem{rum, kirsch, cognac, orange liqueur, coffee, 0.5 Tb}
\end{ingreds}

\recipara{Mleko zagotować na małym ogniu. Ubić żółtka z cukrem 2 -- 3 min do bladości i wstęg.}
\recipara{Wciąż mieszając, bardzo pomału, po kropelkach wlewaj wrzące mleko.}
\recipara{Na mały ogień i podgrzewać do 84 stopni, do bąbelków na brzegach garnka. Ma zgęstnieć na tyle, żeby można było zrobić pasek na łyżce sosu.}
\recipara{Beat in the flavouring.}

\commentskip

\comment{For about 1 cup. Originally twice the amount of sugar, but seemed too sweet.}

\comment{For crème brûlée use the reduced amount of sugar and heavy cream (35\%--40\%) instead of milk. Put in the ramekins with a hot bath and bake in 180 until set edges and jiggly center (25?). To the fridge for at least 4 hours. Compare with some other recipes though.}

\recipend

\section{Crème pâtissière}

\begin{ingreds}
\item{sugar, 1--1.5 Tb}
\item{yolk, 1}
\item{flour 700 2 Tb or 450 1 Tb or starch 0.5 Tb}
\item{milk, 100 g}
\opitem{butter, 1/5 Tb}
\opitem{vanilla, kirsch, cognac, etc.}
\end{ingreds}

\recipara{Boil the milk, beat the sugar with yolks and then beat in the flour. Gradually add the boiling milk.}
\recipara{Put on a moderate heat and stir continuously with a wire whip until gets of desired consistency.}
\recipara{Add in the butter and the flavourings.}

\commentskip

\comment{Homemade ice--cream goes with whites, without flour and butter. 2Tb+ sugar per yolk and 2Tb per white, heated to mix over water bath, no more than 70°C. Beat until fluffy, mix carefully, freeze.}

\comment{For crème Saint--Honoré beat 8 whites per 5 yolks and gradually fold in.}

\recipend

\section{Vegan protein mugcake}

\begin{ingreds}
\item{mleko, 6 łyżek}
\item{płynny słodzik, 2.5 łyżeczki}
\item{mąka, 3 łyżki}
\item{proszek do pieczenia, 0.5 łyżeczki}
\item{białko wegańskie, 2.5 łyżki}
\item{kakao, 2 łyżeczki}
\item{szczypta soli}
\end{ingreds}

\recipara{Dodaj suche składniki do mokrych. Do mikrofalówki na półtorej minuty, potem 15--tki sekund do konsystencji suchawo--gładkiej na górze, nie mokry ani lepki, patyczek suchy albo z małymi okruszkami. W przepisie było 2 łyżki mleka mniej.}

\commentskip

\comment{Można w piekarniku też, w bezpiecznym naczyniu. 5--10 minut w 180 stopniach.}

\comment{Można słodzik w proszku, ale pradwopodobnie potrzeba więcej mleka.}

\recipend

\section{Ciasto daktylowe pełnoziarniste}

\begin{ingreds}
\item{mleko, 1.5 szklanki}
\item{daktyle}
\item{olej, 0.5 szklanki}
\item{mąka pełnoziarnista, 1 szklanka}
\item{proszek do pieczenia, 1 łyżeczka}
\item{soda oczyszczona, 0.5 łyżeczki}
\item{migdały}
\opitem{czarnuszka, 1 łyżeczka}
\opitem{szafran, 1 łyżeczka}
\opitem{imbir, 1 łyżeczka}
\opitem{mak, 1 łyżka}
\end{ingreds}

\recipara{Namocz daktyle w szklance mleka żeby przykryło na 2--3 godziny.}

\recipara{Dodaj pozostałe mleko, olej, zblenduj. Dodaj przesianą mąkę, proszek, sodę, opcjonalne przyprawy. Przelej, posiekaj migdały, posyp na górze.}

\recipara{Piecz 40 minut w 180 stopniach.}

\commentskip

\comment{Konsystencja lejąca, gęsta, lekko utrzymująca się forma. Piekłem 40 w 180, cienka zestalona warstwa z przodu, w środku miłomokro, chociaż nierówno wyrosło (i tak opadło). Pilnuj, żeby dobrze wymieszać proszek i sodę, wtedy może 35? Albo 30 w 190?.}

\comment{Żeby nie opadło, po wyznaczonym czasie wyłącz piekarnik i uchyl tylko lekko drzwiczki, niech stygnie 15 -- 30. Jak dalej opada, może więcej składników zagęszczających (banan?). Może też być zbyt lejące.}

\comment{Banan?}

\recipend

\section{Cynamonowe śliweczki w karmelu Rozkosznego}

\begin{ingreds}
\item{śliwki, 500g}
\item{kwaśna śmietana, 1.5 szklanki}
\item{miód, 2 łyżki}
\item{cynamon, 0.5 łyżeczki}
\item{orzechy laskowe, 0.5 szklanki}
\item{cukier, 0.5 szklanki}
\item{ocet jabłkowy, 2 łyżki}
\item{olej rzepakowy}
\end{ingreds}

\recipara{Kwaśną śmietanę umieścić na gęstym sitku nad miską i wstawić do lodówki do odsączenia na 30min -- 12 hprzed podaniem. Śliwki przekrój na pół, pozbądź się pestek. Orzechy laskowe można zblanszować -- uprażyć na patelni i zetrzeć skórkę ręcznikiem papierowym.}
\recipara{Wysmarować oliwą mały arkusz papieru do pieczenia. Podrzać miód na małej patelni aż zawrze, dodać orzechy laskowe i 0.25 łyżeczki soli. Smażyć 3 minuty. Przełożyć je na natłuszczony papier do pieczenia. Posiekać grubo gdy zupełnie wystygną.}
\recipara{Cukier do średniego garnka, 2 łyżki wody i gotować na średnim $\sim$5min, aż karmel będzie ciemnobursztynowy. Powoli dolewać ocet, następnie śliwki i cynamon. Duś 5min mieszając aż śliwki zmiękną i puszczą sok. Podawać z olejem rzepakowym, z jakiegoś powodu.}

\commentskip

\recipend

\section{Ciecierzycowe ciasteczka z czekoladą erVegan}

\begin{ingreds}
\item{ugotowana ciecierzyca, 1.5 szklanki}
\item{masło orzechowe, 0.5 szklanki}
\item{syrop z agawy, 0.25 - 0.5 szklanki}
\item{soda oczyszczona, 0.25 łyżeczki}
\item{proszek do pieczenia, 0.25 łyżeczki}
\item{czekolada}
\opitem{ekstrakt waniliowy, 1 łyżeczka}
\end{ingreds}

\recipara{Wszystko poza masłęm orzechowym i czekoladą zblenduj na gładką masę. Przełóż do miski, dodaj masło orzechowe i zmieszaj ręcznie. Dodaj pokrojoną w małe kawałki czekoladę i wymieszaj. Nagrzej piekarnik do 180 stopni.}
\recipara{Nabieraj ciasto łyżką i formuj okrągłe ciasteczka lekko spłaszczając je dłońmi. Ułóż na blasze wyłożonej papierem do pieczenia, piecz przez $\sim$10--15min. Powinny się zacząć rumienić, odchodzić od papieru, suchy patyczek. Można trzymać chwilę dłużej żeby były bardziej chrupkie, ale uważać, żeby nie spalić.}

\commentskip

\comment{Oryginalnie w przepisie było blendowane wszystko, ale można rozwalić blender. Spróbuj jak zapisałem.}

\comment{Ważna do blendowania jest płynna konsystencja syropu z agawy; z twardym miodem było trudno.}

\comment{Ekstrakt waniliowy podobno warto.}

\recipend

\section{Mus czekoladowy wegański}

\begin{ingreds}
\item{aquafaba, 150ml, z jednej puszki ciecierzycy}
\item{gorzka czekolada, 150g}
\opitem{świeże owoce}
\opitem{cukier puder, 1 -- 2 łyżki}
\end{ingreds}

\recipara{Czekoladę posiekać i roztopić w kąpieli wodnej.} 
\recipara{Po roztopieniu czekolady wlać aquafabę do miski, dodać szczyptę soli i ubijać mikserem przez 5 minut na sztywną pianę w międzyczasie dodając po łyżce cukru. Dodać przestudzoną czekoladę i delikatnie wymieszać szpatułką.}
\recipara{Przelać do przygotowanych pucharków i wstawić na godzinę do lodówki. Można podawać ze świeżymi owocami.}

\commentskip

\comment{Czekolada powinna wystygnąć do momentu, aż będzie letnia, ale wciąż płynna. Jeśli będzie zbyt ciepła mus nie będzie puszysty; jeśli zbyt zimna, w misce zacznie się łączyć w grudki i zbrylać. Dodanie czekolady od razy po zmiksowaniu to za wcześnie - odczekać jeszcze 10 minut.}
\comment{Gotowy mus może stać w lodówce przez 5 dni.}

\recipend

\section{Lody wegańskie}

\begin{ingreds}
\item{coconut milk, 2 cans, full fat}
\item{sugar, 0.5 cup}
\item{vanilla extract, 1 tsp}
\opitem{flavourings}
\end{ingreds}

\recipara{One can in the fridge overnight. The other can in the pot with the sugar. Bring to light simmer for half to one hour until reduces by about half. Into the fridge as well.}

\recipara{Get the unopened can, carefully scoop the thick cream off the top, beat on high speed until thick and fluffy. Add the condensed milk and continue beating until looks a lot like whipped cream, at the end adding vanilla extract to prevent from getting crystallised.}

\commentskip

\recipend

\section{Rice pudding}

\begin{ingreds}
\item{non-dairy milk, 3 cups}
\item{coconut milk, 1 cup}
\item{rice, 1 cup}
\item{raisins, 0.5 cup}
\item{maple syrup}
\item{vanilla extract, 1 tsp}
\item{orange zest, 1}
\item{cinnammon, 1 tsp}
\item{nutmeg, 0.5 tsp}
\item{vegan butter, 2 tbsp}
\end{ingreds}

\recipara{Stir everything except the butter together, bring to boil, simmer occasionally stirring until the desired thickness, around 20 min. Put in the butter and salt.}

\commentskip

\comment{Ottolenghi serves with browned butter, lime juice and fried rice. First in water for 20 min, then in milk for 30.}

\comment{Also Ottolenghi pęczak-pudding; soak overnight, cook for 1h - when overcooked gets tender.}

\recipend

\section{Sezamowy serek z tofu}

\begin{ingreds}
\item{tofu naturalne, 180g}
\item{mleko roślinne, 0.2}
\item{tahini, 3 łyżki}
\item{wanilia w proszku, 0.3 łyżeczki}
\item{sok z cytryny, 1 łyżka}
\item{banan, 1}
\item{syrop klonowy, 1 łyżeczka}
\item{porzeczki, owoce, 1 szklanka}
\end{ingreds}

\recipara{Zmiksuj wszystko, wymieszaj delikatnie z trzema czwartymi owoców, podawaj z resztą owoców.}

\commentskip

\recipend

\section{Clafoutis}

\begin{ingreds}
\item{duże jajka}
\item{mleko lub śmietana}
\item{mąka pszenna}
\item{cukier}
\item{sól, 0.25 łyżeczki}
\item{szklanka owoców}
\opitem{masło}
\opitem{laska wanilii}
\opitem{skórka cytryny}
\end{ingreds}

\comment{Mleko 1: jajka 3: mąka 0.5: cukier 0.25, ze śmietaną 1:2:0.25:0.125. Jak chcemy gęstszy to +0.5 szklanki śmietany.}

\recipara{Roztrzepać jajka z cukrem, dodaj mleko, sól, dodatki, wymieszaj, potem mąkę. Naczynie natrzyj masłem. Zalej w nim owoce. Piecz aż brzegi zaczną się rumienić, środek może się chwiać, zetnie się przy stygnięciu - 15 minut.}

\recipara{Wykorzystywane do puddingu chlebowego, w małym naczyniu trudniej o porządaną konsystencję. 15 minut w 180 ścina za bardzo brzegi, zostawia środek płynny.}

\commentskip

\comment{Śmietanę można aromatyzować gotując z 4 łodyżkami rozmarynu.}
\comment{Do wersji z mlekiem warto dodać 3 łyżki masła, podsmażonego do bursztynu.}
\comment{Owoce można namaczać w alkoholu, którym potem odpowiednio zastąpić mleko.}

\recipend

\section{Panna Cotta}

\begin{ingreds}
\item{powdered gelatin, 1 tbsp}
\item{cream 36\%, 0.6 l}
\item{whole milk, 360 ml}
\item{sugar, 5 tbsp}
\item{vanilla bean}
\item{ice}
\end{ingreds}

\recipara{In a small saucepan, combine the cream, milk, sugar and split vanilla bean. Bring to a simmer over medium heat.}
\recipara{In the meantime half-fill a large bowl with ice and add enough water to make an ice bath. Set aside. Wipe the insides of 8 120 ml ramekins with a light coating of neutral oil and set aside. As the mixture begins to simmer, place 2 tbsp of water in a small bowl and sprinkle the gelatin over the top. Stir to distribute and set aside to soften (2, 3 min but cannot set).}
\recipara{Remove the mixture from the heat and whisk in the softened gelatin. Scrape the vanilla seeds from the bean pod into the mixture and discard the pod.}
\recipara{Set the saucepan in the ice bath and whisk until the mixture is lukewarm. Rub fingers together: there should be no grit from undissolved gelatin.}
\recipara{Ladle the mixture into the ramekins and fridge-chill for at least 4 hours. About 10 minutes before serving, run a thin-bladed knife around the inside of a ramekin. Dip the ramekin briefly in a bowl of hot tap water and then carefully invert onto a serving plate.}

\commentskip

\comment{If the panna cottas are being chilled for longer than overnight, cover them with a plastic wrap, pressing the wrap gently against the panna cotta to prevent the skin from forming.}
\comment{Preparing the panna cottas more than 24 hours in advance will result in a somewhat firmer set.}

\recipend

\section{G\^{a}teau au chocolat}

\begin{ingreds}
\item{gorzka czekolada, 200g}
\item{masło, 200g}
\item{cukier, 250g}
\item{jajka, 4}
\item{mąka, 80g}
\end{ingreds}

\recipara{Czekoladę z masłem rozpuścić, może być na patelni. Zmieszać z cukrem i ostudzić.}
\recipara{Dodać roztrzepane jajka i mąkę, dokładnie wymieszać. 190 stopni na 20min.}

\commentskip

\recipend

\section{Tiramisu klasyczne}

\begin{ingreds}
\item{biszkopty podłużne, $\sim$200g}
\item{mascarpone, 500g}
\item{jajka, 3 średnie}
\item{cukier lub cukier puder, 50g}
\item{Amaretto, 90ml}
\item{espresso, 250ml}
\item{kakao}
\end{ingreds}

\recipara{Przygotować kawę. Przelać najlepiej do wysokiego naczynia, żeby można było zamoczyć biszkopty. Zostawić do ostygnięcia, dodać Amaretto.}
\recipara{Jajka sparzyć, oddzielić żółtka od jajek. Żółtka utrzeć z cukrem na jasną, puszystą masę. Zmniejszyć obroty na średnie i wmiksować krótko mascarpone. Białka ubić ze szczyptą soli na sztywną pianę. Wmieszać delikatnie szpatułką do masy z mascarpone.}
\recipara{Biszkopty pojedynczo zanurzać w kawie tylko na moment. Ułożyć w formie jeden obok drugiego. Wyłożyć połowę masy, wyrównać. Nałożyć kolejną warstwę biszkoptów i przykryć resztą masy. Wstawić do lodówki na co najmniej 5 godzin.}
\recipara{Przed podaniem sowicie oprószyć kakao.}

\commentskip

\comment{Ostatnie podejśćie wywszło zbyt płynne. Zmniejszyłem ilość jajek, podkreśliłem, by krócej moczyć biszkopty.}

\recipend

\section{Ciasto marchewkowe}

\begin{ingreds}
\item{jajka, 2}
\item{cukier puder, 90g}
\item{olej roślinny, 150ml}
\item{marchew, 200g}
\item{orzechy włoskie, 50g}
\item{ananas (świeży/z puszki), 75g}
\item{wiórki kokosowe, 50g}
\item{mąka, 200g}
\item{proszek do pieczenia, 0.5 łyżeczki}
\item{soda oczyszczona, 1 łyżeczka}
\item{cynamon, 1 łyżeczka}
\item{serek kremowy, np. Philadelphia, 125g}
\item{masło, 50g}
\end{ingreds}

\recipara{Jajka i masło ocieplić w temperaturze pokojowej, marchewkę drobno zetrzeć. Posiekać orzechy i ananasa.}
\recipara{Jajka ubić do podwojenia objętości, dodać 60g cukru i ubijać dalej aż masa będzie gładka i puszysta. Wciąż ubijając na wysokich obrotach dolewać cienkim strumieniem olej.}
\recipara{Dodać marchewkę, ananasa, orzechy, wiórki, delikarnie wymieszać, nagrzać piekarnik do 150 stopni.}
\recipara{W osobnej misce przesiać mąkę, dodać proszek do pieczenia, sodę, cynamon, szczyptę soli, wymieszać. Przesypać do miski z marchewką i delikatnie połączyć składniki.}
\recipara{Ciasto włożyć do formy wyłożonej papierem do pieczenia. Piec do suchego patyczka przez $\sim$1h.}
\recipara{Na polewę ubić serek razem z masłem i resztą cukru pudru. Włożyć na kilkanaście minut do lodówki. Gdy ciasto będzie gotowe i ostygnie, przekroić na pół, środek wysmarować jedną trzecią polewy, resztę dać na górę.}

\commentskip

\recipend

\section{Ciasto pomarańczowe Rozkosznego}

\begin{ingreds}
\item{pomarańcze, 2 małe, 600g}
\item{mąka, 1.75 szklanki}
\item{proszek do pieczenia, 2 łyżeczki}
\item{sól, 0.5 łyżeczki}
\item{kurkuma, 0.5 łyżeczki}
\item{jajka, 3}
\item{cukier, 0.5 szklanki}
\item{olej, 0.5 szklanki}
\item{likier pomarańczowy, 2 łyżki}
\item{ekstrakt waniliowy, 1 łyżka}
\end{ingreds}

\recipara{Jedną pomarańczę gotować przez 60 minut do całkowitej miękkości, wystudzić. Jajka do temperatury pokojowej.}
\recipara{Przekroić wystudzoną pomarańczę, pozbyć się pestek, zblendować. Rozgrzać piekarnik do 190 stopni, keksówkę masłem i mąką. Drugą pomarańczę przekroić na cienkie półksiężyce.}
\recipara{Wymieszać mąkę z proszkiem i kurkumą. Ubić jajka z cukrem puszyście, $2--3$ min, cały czas miksując wlewać strumykiem olej. Dodać puree, likier, ekstrakt, wymieszać do połączenia.}
\recipara{Suche do mokrych i tylko do połączenia. Do keksówki, na wierzchu tyle krojonych pomarańczy ile zapragniesz. Posypać z góry cukrem. Piec około 50 min.}

\commentskip

\comment{To już do zaimprowizowania.}
\comment{Bez likieru i ekstraktu samo ciasto wychodzi trochę ubogie, niech będzie chociaż jedno.}
\comment{Ubiłem białka osobno, średnie pomarańcze, wciąż trochę niewyrośnięte. Kawałki skórki zblendowanej pomarańczy za duże; myślę, że obrać, gotować tylko skórki, potem poszatkować.}
\comment{Z dużymi pomarańczami masa wychodzi dość mokra i w środku ciasta niewyrośnięta. Pilnuj, dostosuj mąkę.}
\comment{Rozkoszny dekoruje cieniutkimi pomarańczami ze skórką zazębiająćymi się na całej długości, wtedy może ok. Jak powsadzałęm tylko kilka to było za dużo proszku do pieczenia.}
\comment{Zamiast proszku do pieczenia powinna starczyć sama soda, wejdzie w reakcję z kwasem pomarańczy. Spróbuj jedną łyżeczkę, 2 za bardzo rozsadziły.}

\recipend

\section{Tofurnik dyniowy z kisielem żurawinowym à la Jadłonomia}

\begin{ingreds}
\item{ciastka oreo, owsiane lub inne, 50g}
\item{nasiona dyni, 100g}
\item{olej kokosowy, 5--6 łyżek}
\item{tofu naturalne, 360g}
\item{ugotowana kasza jaglana, 1 szklanka}
\item{cukier puder, 0.75 szklanki}
\item{dynia, kawałek}
\item{mąka ziemniaczana, 2 łyżki + 1 łyżeczka}
\item{laska wanilii, 1}
\item{pomarańcza, 0.5}
\item{cytryna, 1 średnia}
\item{świeża żurawina lub mrożone maliny, jagody, 0.5 szklanki}
\item{mleko kokosowe, 1 szklanka}
\item{cynamon, 1 łyżeczka}
\end{ingreds}

\recipara{Przygotować puree dyniowe. Rozgrzać piekarnik do 200 stopni, dynię przekroić na połówki lub ćwiartki i wsunąć do piekarnika. Piec do miękkości, $\sim$1 godzinę. Pozbyć się pestek, zmiksować. Będzie potrzebne 0.5 szklanki, resztę można mrozić.}
\recipara{Ugotować kaszę jaglaną, otrzeć skórkę z połowy pomarańczy, wycisnąć sok z cytryny. Potrzebne będzie 0.5 szklanki.}
\recipara{Przygotować spód. Ciastka, nasiona dyni, olej kokosowy i szczyptę soli ziksować w blenderze na grube okruchy. Tortownicę wyłożyć papierem do pieczenia i na dno wysypać pokruszony spód, odstawić do lodówki.}
\recipara{Gdy puree będzie gotowe ochłodzić piekarnik do 180 stopni. Przygotować masę. Umieścić w misce tofu, kaszę, cukier puder, puree, 2 łyżki mąki ziemniaczanej, miąższ z laski wanilii, skórkę pomarańczy, cynamon i szczyptę soli i zblendować ręcznym blenderem na gładką masę. Wówczas stopniowo wlewać sok z cytryny i mleko kokosowe, wciąż blendując.}
\recipara{Na dno piekarnika wstawić naczynie z wodą. Masę wylać na schłodzony spód, wyrównać i wstawić do piekarnika na 15 minut, po czym zmniejszyć temperaturę do 120 stopni i piec kolejne 45 minut.}
\recipara{Wyłączyć piekarnik pozostawiając tofurnik jeszcze na 15 minut w środku. Dopiero potem wyjąć na blat i studzić przez 2--3 godziny.}
\recipara{Po wystudzeniu ciasta przygotować kisiel. W niedużym rondelku zagotować żurawinę z 0.25 szklanki wody i gotować przez 5 minut. W międzyczasie wymieszać łyżeczkę mąki ziemniaczanej z wodą, wlać do garnka z żurawiną i szybko wymieszać. Rozsmarować na wierzchu ciasta.}

\commentskip

\recipend

\section{Pascha wielkanocna}

\begin{ingreds}
\item{świeże mleko, 2 litry}
\item{wanilia, 1 laska}
\item{kwaśna śmietana, 500 ml, co najmniej 18\%}
\item{jajka, 6}
\item{masło, 250g}
\item{cukier puder, 0.75 szklanki}
\item{bakalie}
\item{gaza jałowa}
\end{ingreds}

\recipara{Mleko zagotować z laską. Wyłowić ją, ziarenka wydłubać i pozostawić w mleku. Do gotującego mleka powoli wlewać jajka roztrzepane ze śmietaną. Gotować mieszając, aż mleko w garnku się zetnie na ser i oddzieli serwatka.}
\recipara{Cedzak wyłożyć podwójną warstwą gazy i odcedzić ser. Starannie odcisnąć, zostawić na noc na cedzaku lekko przyciśnięty talerzykiem ze słoikiem z wodą.}
\recipara{Masło zmiksować z cukrem pudrem. Stopniowo dodawać ser i ucierać aż będzie jednolity krem. Dodać bakalie, wymieszać.}
\recipara{Wyłożoną gazą miskę napełnić kremem, wstawić do lodówki. Jak stężeje, przewrócić do góry dnem, zdjąć gazę, udekorować.}

\commentskip

\comment{Wzory i barwniki.}

\recipend
