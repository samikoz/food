\chapter{Zupy}

\section{Wegański bulion}

\begin{ingreds}
\item{zimna woda, 2.5l}
\item{cebula, 2}
\item{marchewka, 3}
\item{pietruszka, 2}
\item{seler, 0.5}
\item{por, zielone liście}
\item{natka pietruszki}
\item{lubczyk, najlepiej świeża łodyżka}
\item{goździki, 2}
\item{ziele angielskie, 4 ziarna}
\item{pieprz, 10 ziaren}
\item{sos sojowy, 2 łyżki}
\item{suszone grzyby, $\sim10$ kawałków}

\end{ingreds}

\recipara{Warzywa poza porem i cebulami nie obrać, wyszorować i w jak największych kawałkach przysmażyć na łyżce oleju w największym garnku. Cebule obrać, zostawić obierki i wnętrza przypalić na palniku albo na blasze piekarnika na najwyższej półce. Liście pora przysmażyć na suchej patelni, polać wódką jak się przybrązowi i odparować.}

\recipara{Do garnka wrzucić łodygi natki pietruszki (liście zostawić), resztę składników bez sosu sojowego i zalać wodą. Podgrzewać pod przykryciem na średnim ogniu do zagotowania. Potem zmniejszyć ogień do minimalnego żeby wywar "mrygał" i zostawić na minimum 3 godziny pod przykryciem. Dobrze zostawić na całą noc już bez ognia.}

\recipara{Rosół przecedzić, a warzywa dokładnie wycisnąć. Posolić, posoić, podawać z siekanymi liśćmi pietruszki.}

\commentskip

\comment{Bulion soli się na końcu, więc i sos sojowy!!}

\recipend

\section{Mercimek çorbası}

\begin{ingreds}
\item{bulion, 1l}
\item{czerwona soczewica, 1 szklanka}
\item{oliwa z oliwek}
\item{cebula, 1}
\item{czosnek, 3}
\item{marchew, 2}
\item{ziemniak, 1}
\item{cytryna, 1}
\item{pasta pomidorowa, 2 łyżki}
\item{piperz Aleppo, 2 łyżeczki}
\item{kmin rzymski, 1 łyżeczki}
\item{mielona kolendra, 0.5 łyżeczki}
\opitem{masło}
\end{ingreds}

\recipara{Posiekaj warzywa poza czosnkiem, przepłucz soczewicę. Rozgrzej oliwę i podsmażaj aż zmiękną, około 5 minut.}

\recipara{Dodaj pastę pomidorową, przemieszaj warzywa. Dodaj bulion, łyżeczkę pieprzu Aleppo, kumin i kolendrę. Doprowadź do wrzenia na $\sim$5 minut, zmniejsz ogień, przykryj zostawiając szparę. Trzymaj na najmniejszym ogniu przez 15 -- 20 minut aż warzywa całkiem zmiękną. Zblenduj.}

\recipara{Czosnek wrzuć na 3 łyżki oliwy z oliwek (opcjonalnie masło) z łyżeczką pieprzu Aleppo, smaż aż się zezłoci. Wykończ nim zupę. Serwuj z kawałkami cytryny na bokach.}

\commentskip

\comment{Ponoć tradycyjnie dodaje się jeszcze biber salçası, pastę z czerwonej papryki.}

\comment{Notorycznie zamiast pasty pomidorowej dodawałem porównywalnie sporą ilość samorobnego sosu z pomidorów. Zanotuj kiedyś ilość ale i spróbuj według przepisu.}

\recipend

\section{Kapuśniak z pomidorami i soczewicą}

\begin{ingreds}
\item{młoda kapusta, 0.5}
\item{cebula, 1}
\item{czerwona soczewica, 0.5 szklanki}
\item{pomidory w puszce, 1}
\item{wędzona papryka, 1 łyżeczka}
\item{ostra papryka, 0.5 łyżeczki}
\item{słodka papryka, 0.5 łyżeczki}
\item{koperek, 1 pęczek}
\item{bulion, 1.25l}
\item{sok z cytryny, 2 łyżeczki}
\end{ingreds}

\recipara{Kapustę posiekać na cienkie paski, głąb zetrzeć na tarce. Cebulę pokroić w kostkę.}
\recipara{W dużym garnku rozgrzać olej i dodać cebulę. Zamieszać, wsypać papryki i szczyptę soli. Smażyć do zeszklenia.}
\recipara{Dodać kapustę, mieszać do utraty objętości, zrobić bulion. Dodać soczewicę, pomidory, dusić przez 5min.}
\recipara{Posiekać koperek, zarówno ogonki jak i puszyste gałązki. Gałązki na bok, ogonki wrzucić do garnka. Wlać bulion, przykryć i gotować $\sim$15min do miękkości soczewicy.}
\recipara{Na koniec wsypać resztę posiekanego koperku, wlać sok z cytryny. Doprawić solą i pieprzem do smaku.}

\commentskip

\recipend

\section{Wegański żurek}

\begin{ingreds}
\item{mąka żytnia razowa}
\opitem{liście laurowe}
\opitem{ziele angielskie}
\opitem{skórka razowego chleba}
\opitem{pieprz czarny ziarnisty}
\item{czosnek}
\item{cebula, 2}
\item{olej, 3 łyżki}
\item{seler, 0.5}
\item{marchew, 1}
\item{ziemniaki, 4}
\item{bulion, 1 litr na 500ml zakwasu}
\item{chrzan, 3 łyżeczki}
\item{majeranek, 1 łyżka}
\end{ingreds}

\recipara{Zrób zakwas na żurek. Do eksperymentacji: szklanka mąki, wymieszać z 1.5 szklanki przegotowanej, ciepłej wody. Dodać jeszcze 1.5 szklanki, skórkę z kromki razowego chleba, 3 ziarenka ziela angielskiego, 3 liście laurowe, 4 ząbki czosnku, 10 ziarenek pieprzu. Przykryć ściereczką, w ciepłe miejsce na 5 dni, mieszając codziennie.}
\recipara{Cebulę, marchew, ziemniaki pokroić, posiekać 4 ząbki czosnku. Smaż cebulę i czosnek na średnim do 5 minut. Seler ze skórki i w drobną kostkę. Dodaj warzywa i smaż przez chwilę.}
\recipara{Wlej bulion, zagotuj i aż warzywa zmiękną. Dodaj chrzan, żur, majeranek, pieprz, sól do smaku. Gotuj jeszcze przez 2 -- 3 minuty.}

\commentskip

\recipend
