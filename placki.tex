\chapter{Placki}

\section{Vegańskie doskonałe pancaki Jadłonomii}

\begin{ingreds}
\item{aquafaba, 4 łyżki}
\item{mleko sojowe, 1.5 szklanki}
\item{ocet, 1 łyżka}
\item{olej, 4 łyżki}
\item{mąka pszenna, 2 szklanki}
\item{proszek do pieczenia, 1.25 łyżeczki}
\item{soda oczyszczona, 0.75 łyżeczki}
\item{cukier puder, 2 łyżki}
\opitem{mąka ziemniaczana, 0.5 łyżeczki}
\end{ingreds}

\recipara{Mleko roślinne połączyć z octem w szklance, zamieszać i zostawić. Obok postawić olej, żeby o nim nie zapomnieć. W misce wymieszać mąkę, proszek do pieczenia, sodę i szczyptę soli.}
\recipara{Do drugiej miski wlać aquafabę i ubijać mikserem aż do momentu, jak będzie sztywna, gęsta i trwała, około 5--7min. Po tym czasie dodać cukier puder, ubijać jeszcze minutę, a na koniec mąkę ziemniaczaną i ostatnią minutę.}
\recipara{Do suchych składników dodać mokre składniki oraz olej. Wymieszać krótko łyżką tak, by skłądniki się połączyły ale wciąż były grudki. Dodać połowę aquafaby, delikarnie wymieszać i to samo zrobić z drugą połową.}
\recipara{Rozgrzać suchą patelnię, wlewać łyżką i smażyć 2--3min z jednej strony, potem 1--2 z drugiej.}

\commentskip

\comment{Cukier puder jest naprawdę ważny.}

\comment{Na własnym mleku owsianym i mące tortowej ciasto wyszło zbyt gęstoklejące, nie smażyło się tak ładnie równo jak na zdjęciach (może głębszy olej?). Spróbować z dwiema szklankami mleka w tym settingu.}
\comment{Na sojowym wyszło przepięknie wstawiając mąkę z mlekiem na noc do lodówki.}

\recipend

\section{Crêpes}

\begin{ingreds}
\item{woda, mleko}
\item{mąka}
\item{masło}
\item{sól}
\item{jajka}
\opitem{cukier}
\opitem{rum}
\end{ingreds}

\recipara{Woda + mleko 1:1 w stosunku do mąki, 1 jajko na pół szklanki mąki, łyżka do półtorej masła.}

\recipara{Do słodkich tylko żółtka, trochę cukru, łyżka rumu. Odstać w lodówce co najmniej 2 godziny.}

\commentskip

\recipend

\section{Dorayaki}

\begin{ingreds}
\item{jajka}
\item{cukier}
\item{miód}
\item{mąka}
\item{mleko}
\end{ingreds}

\recipara{Jajko na 50g mąki, cukier + miód + mleko 1:1 na mąkę, mleka trzy razy mniej.}

\commentskip

\comment{Najpopularnijesze z japońskich słodyczy {\em wagashi}.} 

\comment{Ponoć tradycyjnie podawane z pastą anko, czerwona fasola na słodko.}

\recipend

\section{Naleśniki z okary}

\begin{ingreds}
\item{mąka pszenna razowa, 0.75, 90g}
\item{mąka uniwersalna, 0.5, 60g}
\item{proszek do pieczenia, 1 łyżeczka}
\item{soda oczyszczona, 1 łyżeczka}
\item{cukier, 1 łyżka}
\item{sól, 0.5 łyżeczki}
\item{okara, 1 szklanka, 100g}
\item{gotowane ziarna, 1 szklanka, 150g}
\item{jajko, 1 duże}
\item{mleko sojowe, 1.5, 360ml}
\opitem{masło, 4 łyżki, 60g}
\end{ingreds}

\recipara{1:1 mieszanki mąk z gotowanymi ziarnami, :1 objętościowo z okarą. Jajko na szklankę. Niech odpoczywa co najmniej 10 minut.}

\commentskip

\comment{Ostatnio na żytniej wyszły tłuste. Spróbuj nie dawać tłuszczu ale smażyć na głębokim.}

\comment{Na ziarna dobrze się sprawił bulgur, gorzej kuskus.}

\recipend

\section{Dosa}
\begin{ingreds}
\item{urad dal, 0.5 cup}
\item{rice, 1.5 cup}
\item{methi seeds, 0.5 tsp}
\opitem{poha, 2 tbsp}
\opitem{chana dal, łuskana ciecierzyca, 2 tbsp}
\end{ingreds}

\recipara{Put urad dal, chana dal and methi seeds in one bowl and rice in another. Rinse all thoroughly a few times and soak in filtered water for 6 hours or overnight (in warm climate 4h). 30 min before blending rinse and soak the poha in 0.25 cup filtered water.}
\recipara{Drain the water from dal and pour 0.75 cup of filtered water. Add poha, blend until smooth, frothy and bubbly. Must be thick of pouring consistency. Transfer to another container.}
\recipara{Drain rice and add to blender jar with 0.5 cup water. Blend to slightly coarse batter.}
\recipara{Add the rice to the urad dal, mix both well. Must be of pouring consistency, thick and not runny.}
\recipara{Cover and ferment in a warm place until the batter rises and turns bubbly. In warm regions on the counter overnight, might take 5 -- 16h. In colder regions to the oven on the lowest setting for maybe 8h. To test, half a spoon of batter should float in a bowl of water.}
\recipara{When frying, spread the oil on the pan with a half-cut onion to prevent from sticking.}

\commentskip

\comment{Adding salt to the entire batter shortens its lifespan; should be able to stay in the fridge supposedly for one to two weeks. Using sea salt/Indian rock salt/non-iodized salt helps the fermentation process if added before.}
\comment{High protein dosa has 1 cup rice instead.}
\comment{Can make a celery dosa verion, 4 cups of dosa batter, $\sim$4 sticks of celery, optionally shallots and cilantro. Chop, cook for 4--5 min, drain, let cool, puree and add to dosa batter.}

\comment{Can try with other proteins. Went with groch, 1:1:0.5 with water and rice flour; should go for more groch.}

\recipend

