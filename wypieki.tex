\chapter{Wypieki}

\section{Drożdżowe}

\begin{ingreds}
\item{mąka, 2, \textit{ponoć najlepsza 550}}
\item{mleko, 0.5}
\item{drożdże, pół suchych, 25 świeżych}
\item{jajko}
\item{tłuszcz, 50g}
\item{cukier, 4 łyżki}
\end{ingreds}

\recipara{Rozczyn z połowy mąki i pół łyżki miodu, ma się podwoić. Wyrabiać około 15 min, tłuszcz na końcu. Ma się podwoić. Potem odgazować, chwilę wyrabiać, uformować, znowu podwoić.}

\commentskip

\comment{Do mąki wagowo mleko 1:2, tłuszcz cukier 1:6.}

\comment{Temperatura zaczynu dla świeżych 35, suszonych 45.}

\comment{Shokupan: mąka 700, 8\% Tangzhong (mąka woda/mleko 1:5; nie grzej do końca, ma być lejące), cukier 10\%, masło 1:12, sól 2\%, drożdży trochę mniej, jajka pół. Na filmikach są gładsze i bardziej lśniące, można dodawać śmietanę wcześniej, wodę na końcu też.}

\comment{Racuchy: mleko 1, 2--3 jabłka, szklanka cukru; jagodzianki Rozkosznego 0.25.}

\comment{Chałka: mąka chlebowa, woda, 20g cukru, szczypta soli, 2 żółtka.}

\recipend

\section{Kruche}

\subsection{Pâte sablée}

\recipara{Delikatne i kruchliwe, ale chrupkie, do słodkich tartaletek.}

\begin{ingreds}
\item{flour all-purpose, 500-550, 2}
\item{sugar, 3--7 Tb}
\item{baking powder, 1/8}
\item{butter, 7 Tb}
\item{egg, 1}
\item{vanilla, 0.5 tsp}
\end{ingreds}

\recipara{Ubij jajko z łyżką wody. Schłodź masło.}

\recipara{Mąka, cukier, masło, proszek do miski. Prędko ugniataj palcami aż tłuszcz rozpadnie się do kawałków rozmiarów płatków owsianych. Dodaj ubite jajko i szybko wyrabiaj, ciasto będzie mocno kleiste. Na koniec wgnieć nieco ciasto i rozciągnij na 6 cali zabierając z powrotem. Zawiń w papier do pieczenia i schłodź.}

\recipara{Wałkuj między dwoma papierami do pieczenia. Piecz w 180 z obciążeniem 5--6 minut, potem podziurkuj i znowu aż zacznie się delikatnie rumienić, 8--10.}

\commentskip

\comment{Wyszło trochę zbyt brittle, ale była mąka 700. Im więcej cukru tym kruchsze, ale ciasto lepksze. Spokojnie spróbuj z połowy przedziału.}

\comment{Piekłem z grzaniem na dole, najpierw 5--6, ale potem dłużej, koło 20. Trudno było wyłuskać z tartowego naczynia, d27cm, połamało się na trzy. Wydaje się, że lepiej w tortownicy z usuwanymi ściankami.}

\recipend

\section{Lembasy chia}

\begin{ingreds}
\item{chia, 0.5 szklanki}
\item{pestki słonecznika, 0.5 szklanki}
\item{pestki dyni, 0.5 szklanki}
\item{płatki owsiane zmielone, 0.5 szklanki}
\item{kasza gryczana niepalona zmielona, 0.25 szklanki}
\item{woda, 1.3 szklanki}
\item{oliwa, 0.5 łyżki}
\item{oregano, 1 łyżka}
\item{tymianek, 0.5 łyżki}
\item{sól, 0.5 łyżki}
\item{czosnek, 0.25 łyżki}
\item{cukier, 0.5 łyżki}
\end{ingreds}

\recipara{180 stopni, 30 minut.}

\commentskip

\recipend

\section{Musztardowy chleb żytni à la Justyna Dragan}

\begin{ingreds}
\item{aktywny zakwas żytni, 200g}
\item{letnia woda, 400ml}
\item{miód, 20g}
\item{musztarda, 100g}
\item{mąka żytnia 720, 200g}
\item{mąka żytnia razowa 2000, 250g}
\item{siemię lniane}
\item{słonecznik łuskany}
\item{sól, 10g}
\end{ingreds}

\recipara{Wszystkie składniki umieścić w misce i wymieszać tylko do połączenia. Przykryć ściereczką i zostawić na ~4h w ciepłym piekarniku, aż ciasto urośnie i będzie miało napowietrzoną strukturę.}
\recipara{Mniejszą keksówkę natłuścić oliwą i solidnie obsypać mąką żytnią razową. Ciasto do formy, wyrównać powierzchnię zwilżoną w wodzie łyżką. Posypać słonecznikiem, łyżką delikatnie wbić go w ciasto. Przykryć na ~4h w ciepłym piekarniku do momentu, aż ciasto wcisniętę będzie powoli wracać do pozycji początkowej.}
\recipara{Rozgrzać piekarnik do 220 stopni, przykryć chleb folią aluminiową. Piec 45 góra-dół. Wyjąć blaszkę z wodą, zdjąć folię, formę obrócić, piec 15 minut.}
\recipara{Chleb wyjąc z keksówki i umieścić na kratce do studzenia. Kroić po całkowitym wystudzeniu, najlepiej po nocy, żeby w środku również się wystudził.}

\commentskip

\comment{Dokarmianie zakwasu, regularnie 1:1:0.9 i co 4--6 godzin; jak na całą noc to $\sim$x5 więcej, ale tak jeszcze nie wyszło mi x3.}

\comment{Faza wyrastania zimą: około 16h w lodówce, 4.5h na kaloryferze. Przy miksie lodówka + kaloryfer wychodzi pierw łagodniejszy, a potem pojawia się kwasek. Ciekawe czy przez kaloryfer.}

\comment{Blaszka z wodą uniemożliwia żądanego nagrzania piekarnika; wtedy miał 160 i piekłem pod folią 1h10. Ponoć i tak nie ma znaczenia, co najwyżej spryskaj trochę piekarnik na początku.}

\comment{Obecne ilości za duże na moją keksówkę. Spróbuj x0.9.}

\recipend

\section{Souffl\'e}

\recipara{Suflet wytrawny na jedną robi się z gęstego beszamelu na 0.25 szklanki mleka (0.75 łyżki masła i mąki) i dodaje 0.25 szklanki sosu smakowego. Więcej białek wtedy, 1.5 żółtka na 2 białka.}
\recipara{Suflet słodki robi się z sosu z cukrem zamiast masła, mąkę najpierw roztapia w odrobinie mleka potem reszta i aż zgęstnieje; 1 żółtko na 1.5 białka. 0.5 łyżeczki ekstraktu waniliowego.}
\recipara{Czekoladowy zamiast mąki skrobia kukurydziana lub 0.5 łyżki ziemniaczanej. Zmieszać z mlekiem (pierw mało, potem resztę) i cukrem gotować tylko 3 sekundy. 1.5 żółtka i 2 białka (potwierdź).}
\recipara{Nagrzać piekarnik do 200, potem zmniejszyć do 190. Wytrawne 25, słodkie 15 minut. Cebulowy był też 15, ale cebula trochę rozwodniona.}

\subsection{pomarańczowy}\vspace{-1em}
\recipara{1 łyżka likieru pomarańczowego. Cukrem natrzeć pomarańczę.}

\subsection{kawowy}\vspace{-1em}
\recipara{1 łyżka kawy.}

\subsection{czekoladowy}\vspace{-1em}
\recipara{35g czekolady (ostatnio 30 było za mało; chyba, że żółtko na 1.5 białka), rozpuścić z niepełną łyżką kawy, dodać pół łyżki masła. Przygotowując bazę ostrożnie i mieszaj bardzo, robią się grudki. Piekłem w 200 15 i troszkę za długo.}

\commentskip

\comment{W ilościach na 4 osoby pieką się 30 minut, czekoladowy 40.}

\recipend

\section{Faworki}

\begin{ingreds}
\item{mąka, 400}
\item{żółtka, 4}
\item{cukier puder, 1 łyżka}
\item{śmietana, 0.5 szklanki}
\item{spirytus, wódka, 2 łyżki}
\item{głęboki tłuszcz}
\item{cukier puder}
\end{ingreds}

\recipara{Wszystko do połączenia, do folii i w lodówce przez pół godziny.}
\recipara{Wyrabiać starając się wtłoczyć jak najwięcej powietrza; uderzać wałkiem, rolować na placek i zawijać na pół.}
\recipara{Rozwałkować jak najcieniej, pociąć na paseczki, potem na romby, zrobić szparę w środku i przeciągnąć przez nią jeden koniec.}
\recipara{Smażyć na głębokim tłuszczu i wykładać na papier. Podawać z cukrem pudrem.}

\commentskip

\comment{Alkohol hamuje absorpcję tłuszczu podczas smażenia.}
\comment{Smażyć raczej do bladobrązowych, stosunkowo łatwo przesadzić. Wysmakować kilka pojedynczo. Doczytać o temperaturze oleju. Chcemy by utrwaliły się bąbelki na powierzchni.}
\comment{Robiłem na 500g mąki szklankę, 190g samorobnej śmietany sojowej w proporcji 1:0.6. Wyszło za tłuste, ciasto się rozwarstwiało, nie chciało lepić, wychodził tłuszcz pod palcami.}

\recipend
