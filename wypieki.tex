\chapter{Wypieki}

\section{Drożdżowe}

\begin{ingreds}
\item{mąka, 2}
\item{mleko, 0.5}
\item{drożdże, pół suchych, 25 świeżych}
\item{jajko}
\item{tłuszcz, 50g}
\item{cukier, 4 łyżki}
\end{ingreds}

\recipara{Rozczyn z połowy mąki i pół łyżki miodu, ma się podwoić. Wyrabiać około 15 min, tłuszcz na końcu. Ma się podwoić. Potem odgazować, chwilę wyraciać, uformować, znowu podwoić.}

\commentskip

\comment{Mąka ponoć najlepsza 550; można ogrzać.}

\comment{Temperatura zaczynu dla świeżych 35, suszonych 45.}

\comment{Spróbuj niedbałej tureckiej zawijki.}

\comment{Na racuchy szklanka mleka, 2--3 jabłka, szklanka cukru; jagodzianki Rozkosznego 0.25.}

\recipend

\section{Lembasy chia}

\begin{ingreds}
\item{chia, 0.5 szklanki}
\item{pestki słonecznika, 0.5 szklanki}
\item{pestki dyni, 0.5 szklanki}
\item{płatki owsiane zmielone, 0.5 szklanki}
\item{kasza gryczana niepalona zmielona, 0.25 szklanki}
\item{woda, 1.3 szklanki}
\item{oliwa, 0.5 łyżki}
\item{oregano, 1 łyżka}
\item{tymianek, 0.5 łyżki}
\item{sól, 0.5 łyżki}
\item{czosnek, 0.25 łyżki}
\item{cukier, 0.5 łyżki}
\end{ingreds}

\recipara{180 stopni, 30 minut.}

\commentskip

\recipend

\section{Musztardowy chleb żytni à la Justyna Dragan}

\begin{ingreds}
\item{aktywny zakwas żytni, 200g}
\item{letnia woda, 400ml}
\item{miód, 20g}
\item{musztarda, 100g}
\item{mąka żytnia 720, 200g}
\item{mąka żytnia razowa 2000, 250g}
\item{siemię lniane}
\item{słonecznik łuskany}
\item{sól, 10g}
\end{ingreds}

\recipara{Wszystkie składniki umieścić w misce i wymieszać tylko do połączenia. Przykryć ściereczką i zostawić na ~4h w ciepłym piekarniku, aż ciasto urośnie i będzie miało napowietrzoną strukturę.}
\recipara{Mniejszą keksówkę natłuścić oliwą i solidnie obsypać mąką żytnią razową. Ciasto do formy, wyrównać powierzchnię zwilżoną w wodzie łyżką. Posypać słonecznikiem, łyżką delikatnie wbić go w ciasto. Przykryć na ~4h w ciepłym piekarniku do momentu, aż ciasto wcisniętę będzie powoli wracać do pozycji początkowej.}
\recipara{Rozgrzać piekarnik do 220 stopni umieszczając na dole blaszkę z gorącą wodą. Przykryć chleb folią aluminiową. Piec 40 na dole piekarnika. Wyjąć blaszkę z wodą, zdjąć folię, formę obrócić, piec 20 minut. Wyjąć chleb z formy i przypiekać jeszcze od dołu przez 5 minut.}
\recipara{Chleb wyjąc z keksówki i umieścić na kratce do studzenia. Kroić po całkowitym wystudzeniu, najlepiej po nocy, żeby w środku również się wystudził.}

\commentskip

\recipend

\section{Aromatyczny chleb z podsmażaną cebulą Justyny Dragan}

\begin{ingreds}
\item{zakwas żytni, 65g}
\item{mąka pszenna 650, 100g + 450g}
\item{mąka żytnia razowa 2000, 100g}
\item{ciepła woda, 60ml + 300ml}
\item{cebula, 2 duże}
\item{drożdże świeże, 6g}
\item{sól, 2 łyżeczki}
\item{olej rzepakowy, 2 łyżki}
\item{melasa lub miód, 1 łyżeczka}
\item{kminek mielony lub kmin rzymski, 1.5 łyżeczki}
\end{ingreds}

\recipara{Wieczór wcześniej do zakwasu wlać 60ml wody i 100g mąki pszennej.}
\recipara{Cebule zeszklić, zarumienić, wystudzić. Obydwie mąki z wodą, solą i przyprawą. W ciepłej wodzie rozpuścić drożdże i melasę. Wlać do mąk, dodać zaczyn oraz olej. Wyrabiać ciasto do elastyczności, nie powinno się kleić zbytnio do dłoni.}
\recipara{Przełożyć do dużej, natłuszczonej miski, przykryć folią i zostawić w ciepłym piekarniku 1 -- 2h do podwojenia objętości. Dobrze jest w trakcie wyrastania odgazować ciasto uderzając w nie pięścią.}
\recipara{Zagnieść i złożyć jeszcze parę razy, uformować bohenek, oprószyć mąką i na blachę z papierem do pieczenia. Odstawić na 40 -- 60 min do napuszenia w ciepłym piekarniku.}
\recipara{Spryskać wodą i piec w 230 stopniach przez 20 minut, potem w 200 przez 7 -- 10. Aż uderzony od spodu wyda głuchy odgłos.}

\commentskip

\comment{300ml jest ok, zrobi się mokrzejsze wraz z wyrabianiem.}

\recipend

\section{Souffl\'e}

\recipara{Suflet wytrawny na jedną robi się z gęstego beszamelu na 0.25 szklanki mleka (0.75 łyżki masła i mąki) i dodaje 0.25 szklanki sosu smakowego. Więcej białek wtedy, 1.5 żółtka na 2 białka.}
\recipara{Suflet słodki robi się z sosu z cukrem zamiast masła, mąkę najpierw roztapia w odrobinie mleka potem reszta i aż zgęstnieje; 1 żółtko na 1.5 białka. 0.5 łyżeczki ekstraktu waniliowego.}
\recipara{Czekoladowy zamiast mąki skrobia kukurydziana lub 0.5 łyżki ziemniaczanej. Zmieszać z mlekiem (pierw mało, potem resztę) i cukrem gotować tylko 3 sekundy. 1.5 żółtka i 2 białka (potwierdź).}
\recipara{Nagrzać piekarnik do 200, potem zmniejszyć do 190. Wytrawne 25, słodkie 15 minut. Cebulowy był też 15, ale cebula trochę rozwodniona.}

\subsection{pomarańczowy}\vspace{-1em}
\recipara{1 łyżka likieru pomarańczowego. Cukrem natrzeć pomarańczę.}

\subsection{kawowy}\vspace{-1em}
\recipara{1 łyżka kawy.}

\subsection{czekoladowy}\vspace{-1em}
\recipara{35g czekolady (ostatnio 30 było za mało; chyba, że żółtko na 1.5 białka), rozpuścić z niepełną łyżką kawy, dodać pół łyżki masła. Przygotowując bazę ostrożnie i mieszaj bardzo, robią się grudki. Piekłem w 200 15 i troszkę za długo.}

\commentskip

\comment{W ilościach na 4 osoby pieką się 30 minut, czekoladowy 40.}

\recipend
