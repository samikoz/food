\chapter{Inne}

\section{Wegańskie masło}

\begin{ingreds}
\item{mleko roślinne, 80ml}
\item{lecytyna słonecznikowa, 4 łyżeczki}
\item{olej roślinny, 1 łyżki}
\item{kiszonka, 2 łyżeczki}
\item{pasta miso, 0.5 łyżeczki}
\item{ocet jabłkowy, 0.25 łyżeczki}
\item{olej kokosowy, 160g, $\sim$8 łyżek}
\item{sól}
\end{ingreds}

\recipara{Zblendować wszystko poza olejem kokosowym i solą. Dodać olej, zblendować. Dodać sól do smaku.}

\commentskip

\comment{Przepis na podwójną ilość, oryginalnie alternatywa miso i ocet albo kiszonka.}

\recipend

\section{Makaron}

\begin{ingreds}
\item{mąka 00 lub semolina ew krupczatka ew mieszanka}
\item{woda}
\opitem{jajka}
\end{ingreds}

\recipara{110 g ciepłej wody na 200 g mąki 00. Wyrabiać około 15 minut, zostawić w folii w lodówce jak najdłużej, minimum 30 minut.}

\recipara{Gotować w osolonej około 90s.}

\commentskip

\comment{Zamiast odmierzać wodę, wlej do wulkanika. Niech się nie rwie. Na cienkie paseczki a potem zroluj do grubego spaghetti.}

\comment{Przyjmuje się jedno jajko na 100g mąki, ale same jajka na mąkę 00 nie są wyjątkowe i ciasto trochę się rwało. Gotowało się wtedy $\sim$10min!. Próbuj z żółtkami.}

\comment{Olej ułatwia wyrabianie ale przeszkadza glutenowi.}

\comment{Czas w folii służy nawodnieniu ciasta.}

\recipend

\section{Majonez wegański}

\begin{ingreds}
\item{aquafaba, 125ml}
\item{płatki drożdżowe, 1 łyżka}
\item{ocet lub sok z cytryny, 2 łyżeczki}
\item{musztarda, 0.75 łyżeczki}
\item{sól, 0.5 łyżeczki}
\item{syrop z agawy lub inny słodzik, 0.25 łyżeczki}
\item{olej roślinny, 1.25 -- 2 szklanki}
\end{ingreds}

\recipara{Wszystko poza olejem włożyć do blendera i miksować przez 1 -- 2 minuty na najwyższych obrotach do czasu, aż porządnie się spienią.} 
\recipara{Wlać olej cienkim strumieniem podczas gdy blender jest włączony. W połowie wlewania oleju zacząć mocno ruszać blenderem góra -- dół i lać dalej olej. Majonez będzie gotowy gdy zgęstnieje.}

\commentskip

\comment{1.25 szklanki oleju daje majonez trochę zbyt płynny.}
\comment{Majonez może stać w lodówce do 14 dni.}

\recipend

\section{Mleka wegańskie}
\recipara{Wody powinnny być filtrowane. Zawsze odsącz, wypłucz, blenduj, odcedź. Czasy moczenia jeszcze przetestuj.}

\subsection{sojowe}
\recipara{0.5 szklanki soi, 2--3 szklanki do namaczania, 4 do blendowania. Przez noc, zagotuj przez 30.}

\subsection{migdałowe}
\recipara{szklanka na dwie wody. Przez noc lub 24h.}

\subsection{nerkowcowe}
\recipara{1:4, przez noc.}

\subsection{laskowe}
\recipara{100g na dwie wody.}

\subsection{kokosowe}
\recipara{1.5--2 szklanek wiórek lub chipsów na 4 woddy. Wodę podgrzej do parowania ale nie wrzenia, wrzuć wszystko do blendera do gęstej i kremowej.}

\subsection{ryżowe}
\recipara{Albo szklanka gotowanego na 4 wody, blenduj od razu. Albo surowego w 1.25 wody przez noc i wtedy 4 szklanki.}

\commentskip

\recipend

\section{Śmietana sojowa}

\begin{ingreds}
\item{mleko sojowe, 0.5}
\item{olej, 0.3 - 0.5}
\item{kwasek cytrynowy, 0.3 łyżeczki}
\item{ciepła woda, 1 łyżka}
\item{sól}
\end{ingreds}

\recipara{Kwasek do wody, do mleka, ubijaj aż pojdzie piana. Dodawaj olej i miksuj.}

\recipara{Gęstnieje po parunastu minutach w lodówce.}

\commentskip

\comment{Nawet 0.3 oleju daje mocnotłustą śmietanę. Próbuj mniej, można eksperymentować z substancjami żelującymi aby osiągać konsystencję.}

\comment{Sok z cytryny zamiast kwasku daje zbyt wyraźny cytrynowy posmak.}

\recipend

