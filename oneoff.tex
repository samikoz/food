\documentclass[11pt]{report}
\usepackage[utf8]{inputenc}
\usepackage[MeX]{polski}
\usepackage[a4paper, left=3.25cm, right=2.5cm, top=2.5cm, bottom=2.5cm, headsep=1.2cm]{geometry}
\usepackage{titlesec}
\titleformat{\chapter}[display]{\bfseries\centering}{}{0pt}{\Huge}[\newpage]
\titleformat{\section}[display]{\normalfont\bfseries}{}{0pt}{\hspace{-0.75cm}\huge}
\titlespacing{\section}{0pt}{50pt}{\parskip}
\titleformat{\subsection}[block]{\normalfont\bfseries}{}{0pt}{}
\usepackage{enumitem}
\newlist{ingreds}{itemize}{1}
\setlist[ingreds]{label=$\bullet$, leftmargin=*}

\usepackage{tocloft}
\renewcommand{\contentsname}{}
\renewcommand{\numberline}[1]{}
\setlength{\cftbeforetoctitleskip}{-1cm}

\newcommand\opitem{\item[\textbf{$\ast$}]}
\newcommand\recipara[1]{#1\par}
\newcommand\commentskip{\vspace{0.75cm}}
\newcommand\comment[1]{\textit{#1}\par}
\newcommand\recipend{\newpage}

\setlength{\parindent}{0pt}
\setlength{\parskip}{1em}
\begin{document}

\section{Cynamonowe śliweczki w karmelu Rozkosznego}

\begin{ingreds}
\item{śliwki, 500g}
\item{kwaśna śmietana, 1.5 szklanki}
\item{miód, 2 łyżki}
\item{cynamon, 0.5 łyżeczki}
\item{orzechy laskowe, 0.5 szklanki}
\item{cukier, 0.5 szklanki}
\item{ocet jabłkowy, 2 łyżki}
\item{olej rzepakowy}
\end{ingreds}

\recipara{Kwaśną śmietanę umieścić na gęstym sitku nad miską i wstawić do lodówki do odsączenia na 30min -- 12 hprzed podaniem. Śliwki przekrój na pół, pozbądź się pestek. Orzechy laskowe można zblanszować -- uprażyć na patelni i zetrzeć skórkę ręcznikiem papierowym.}
\recipara{Wysmarować oliwą mały arkusz papieru do pieczenia. Podrzać miód na małej patelni aż zawrze, dodać orzechy laskowe i 0.25 łyżeczki soli. Smażyć 3 minuty. Przełożyć je na natłuszczony papier do pieczenia. Posiekać grubo gdy zupełnie wystygną.}
\recipara{Cukier do średniego garnka, 2 łyżki wody i gotować na średnim $\sim$5min, aż karmel będzie ciemnobursztynowy. Powoli dolewać ocet, następnie śliwki i cynamon. Duś 5min mieszając aż śliwki zmiękną i puszczą sok. Podawać z olejem rzepakowym, z jakiegoś powodu.}

\commentskip

\recipend
\end{document}
