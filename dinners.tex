\chapter{Dania główne}

\section{Ratatouille}

\begin{ingreds}
\item{aubergines, 250g}
\item{courgettes, 250g}
\item{onions, 250g}
\item{green peppers, 2}
\item{garlic, 2 cloves}
\item{tomatoes, 0.5kg}
\item{parsley, 3 tbl}
\end{ingreds}

\recipara{Peel the aubergines and cut into long slices, thick enough to retain juices and for easy side--changing on a pan. Scrub the courgette and do the same. Place in a bowl, toss with salt, let stand for half an hour. Drain, dry each slice with a cloth.}
\recipara{Skin tomatoes and gently squeeze them over a sieve to get rid of the seeds and juices, keep the juices. Dice the onions and peppers, press the garlic.}
\recipara{One layer at a time, sauté the aubergines and then the courgettes in olive oil for about a minute each side to brown lightly. Remove to side dish. In the same pan cook onions and peppers in olive oil, slowly, until tender not browned. Stir in the garlic, salt and pepper.}
\recipara{Slice the tomato pulp into strips, lay them over the onions and peppers. Salt and pepper, cover, cook over low heat for 5 min or until tomatoes start to yield juices. Uncover, baste with juices, raise heat and boil for several minutes until the juices have almost evaporated.}
\recipara{In a casserole place a third of the tomato mixture, then parsleym half of aubergines and courgettes, half of remaining tomatoes, parsley, rest of aubergettes, rest of tomatoes and finish with parsley. Cover the casserole and simmer over low heat for 10 min. Uncover, tip the casserole gathering juices and baste with them. Slightly raise heat and cook uncovered for 15 minutes more, basting several times until all the juices are gone except a spoonful or two. Do not let the vegetables scorch at the bottom of the casserole.}

\commentskip

\recipend

\section{Caramelized Onion Pasta à la Rainbow Plant Life}

\begin{ingreds}
\item{onions, 2 medium}
\item{garlic, 4 cloves}
\item{red pepper flakes}
\item{dried oregano, 1tsp}
\item{tubed tomato paste, 150g}
\item{tamari or soy sauce, 2tbsp}
\item{nutritional yeast, 2tbsp}
\item{sundried tomatoes, 4--5 large}
\item{spaghetti, 285g}
\opitem{toasted pine nuts}
\opitem{parsley}
\end{ingreds}

\recipara{Slice the onions as thin as you can. Heat 2 tablespoons of olive oil on a pan on a medium--high to high heat and put the onions seasoning with salt and pepper. Stir frequently until light brown fond starts to appear and deglaze. Continue cooking and stirring, deglazing every couple of minutes to prevent burning. After $\sim$15mins the onions should be deeply golden brown. Slice the garlic, put the water for pasta on fire.}
\recipara{Reduce the heat to medium, stir in the garlic cloves, stir frequently adding water if necessary. Cook for $\sim$5--10mins until the onions are fully caramelized and the garlic is golden and soft. Midway through put the pasta into the boiling water. Later reserve a bit of water. Chop the tomatoes.}
\recipara{Add the red pepper flakes and oregano, fry for about 30 seconds. Add the tomato paste, stirring constantly until the paste becomes darker, for 2--3mins. Add the tamari, nutritional yeast and chopped tomatoes. Turn the heat to low.}
\recipara{Ladle a few spoons of pasta water, then transfer the pasta to the onions and cover it with sauce, adding more water if necessary though the texture should stay pretty thick.}

\commentskip

\comment{Umami explosion.}

\recipend

\section{Shepherd's pie à la Jadłonomia}

\begin{ingreds}
\item{namoczona we wrzątku zielona soczewica, 400g}
\item{gęsty przecier pomidorowy, 1 szklanka}
\item{średnie ziemniaki, 5}
\item{nieduże marchewki, 2}
\item{seler naciowy, 2 łodygi}
\item{duża cebula, 1}
\item{mleko, 0.3 szklanki}
\item{tymianek, szczypta}
\item{rozmaryn, szczypta + szczypta}
\item{gałka muszkatołowa, szczypta}
\item{czosnek, 2 ząbki}
\item{masło roślinne, 3 łyżki + kilka płatków}
\end{ingreds}

\recipara{Soczewicę namoczyć przez 1--2h. Opłukać i gotować w świeżej wodzie do miękkości. Powinna być bardzo miękka, ale się nie rozpadać. Z reguły wystarczy $\sim$20min. Obrać ziemniaki, przekroić na połówki i gotować do miękkości w osolonej wodzie.}
\recipara{Warzywa i jeden z ząbków czosnku pokroić w kostkę, rozgrzać kilka łyżek oleju na dnie dużego garnka i podsmażyć cebulę. Po zeszkleniu dodać marchew i seler naciowy, smażyć przez $\sim$5--8min na małym ogniu delikatnie mieszając.}
\recipara{Odcedzić soczewicę, dodać do podsmażonych warzyw, wymieszać, dodać koncentrat, wycisnąć drugi ząbek czosnku, dodać sól, pieprz. Dusić na małym ogniu przez $\sim$10--15min. Spróbować, doprawić solą i pieprzem jeśli trzeba.}
\recipara{Odcedzić ziemniaki, dodać mleko, łyżki masła, gałkę, sól i pieprz i rozgnieść na gładkie puree.}
\recipara{Piekarnik rozgrzać do 200 stopni. Naczynie żaroodporne wysmarować olejem, wyłożyć soczewicową potrawkę, na nią łyżkami wykładać puree, delikatnie rozsmarowując. Posypać rozmarynem, na ziemniakach ułożyć płaty zimnego masła i zapiekać przez $\sim$20--30min.}

\commentskip

\comment{Sara: do potrawki soczewicowej dodać trochę cukru, soku z cytryny/octu jabłkowego/winnego.}
\comment{Sara: jeśli pomidory/przecier nie są bardzo pomidorowe, dodać keczupu/koncentratu.}

\recipend

\section{Chilli sin carne}

\begin{ingreds}
\item{czerwona fasola, 4 puszki/2 szklanki}
\item{krojone pomidory, 2 puszki}
\item{seler naciowy, 2 laski}
\item{czerwona cebula, 2}
\item{czerwona papryka, 2}
\item{marchew, 1}
\item{sos sojowy, 2 łyżki}
\item{ocet balsamiczny, 1 łyżka}
\item{kmin mielony, 1.5 łyżeczki}
\item{ostra papryka, 1.5 łyżeczki}
\item{słodka papryka, 1 łyżeczka}
\item{wędzona papryka, 1 łyżeczka}
\item{cynamon, 0.5 łyżeczki}
\item{świeża kolendra}
\item{limonka, 1}
\opitem{gorzka czekolada}
\end{ingreds}

\recipara{Jeżeli używamy suchej fasoli to namoczyć i ugotować.}
\recipara{Fasolę odcedzić i opłukać, jeśli używamy konserwowej. Cebulę, seler, paprykę i marchew pokroić w kostkę. W dużym garnku rozgrzać olej, dodać warzywa i przyprawy, dusić na małym ogniu przez $\sim$15min.}
\recipara{Dodać fasolę i pomidory, sos sojowy, ocet, sól i pieprz. Wymieszać i dusić przez godzinę. Zdjąć przykrycie jeśli sos jest za rzadki. Doprawić pod koniec, jeśli trzeba.}
\recipara{Podawać z ryżem lub pieczywem, pokropioną limonką i kolendrą.}

\commentskip

\comment{Poeksperymentować z gorzką czekoladą w połowie gotowania. 20 gramów nie daje wyczuwalnego efektu.}
\comment{Można spróbować dać więcej selera.}

\recipend

\section{Risotto gorgonzola pere noci à la Samuelle}

\begin{ingreds}
\item{ryż Carnaroli, 160g}
\item{bulion, z 1 kostki}
\item{gruszka, 1}
\item{czerwona cebula, 0.5}
\item{gorgonzola, 50g}
\item{parmezan tarty, 50g}
\item{białe wino, 0.5 kieliszka}
\item{orzechy włoskie}
\item{masło, do smażenia}
\end{ingreds}

\recipara{Połowę obranej gruszki pociąć w kostkę, drugą połowę zgnieść. Cebulę posiekać, gorgonzolę pociąć na kawałki.}
\recipara{Cebulę smażyć na małym ogniu aż wystarczająco zwiędnie, w tym czasie zrobić bulion. Nie wyłączać ognia, niech wrze.}
\recipara{Do cebuli dosypać ryżu, mieszać podsmażając aż będzie dostatecznie miękki (spróbuj). Wlać wino, trzymać na ogniu aż odparuje.}
\recipara{Dodawać po trochu bulionu do wsiąknięcia. 5 minut przed końcem dodać zgniecioną gruszkę (też zawiera wodę), gorgonzolę i parmezan.}
\recipara{Podawać z orzechami, drugą połową gruszki i parmezanem.}

\commentskip

\recipend

\section{Dynia z kaszą aromatyczną}

\begin{ingreds}
\item{dynia, 1}
\item{orzechy laskowe lub włoskie}
\item{kasza kuskus lub jaglana}
\item{cynamon lub garam masala}
\item{goździki}
\item{gałka muszkatołowa}
\item{liście laurowe}
\item{ziele angielskie}
\opitem{imbir}
\opitem{żurawina jeśli orzechy włoskie}
\end{ingreds}

\recipara{Hokkaido lepiej obrać ze skórki, piżmowa zdaje się nie robić różnicy. Dynię pokroić raczej drobno, zamarynować w oliwie, z gałką, solą i imbirem. Upiec wraz z orzechami laskowymi, je przez 15 min.}
\recipara{Jeśli kasza jaglana to szklanka suchej na jedną dynię; kuskus chyba wystarczy pół. Wodę odpowiednią do zagotowania kaszy zagotować z laską cynamonu, goździkami, liściem laurowym i zielem angielskim. Jeśli cynamon nie w lasce to najpierw podsmażyć. Gotować przez minutę i zalać kaszę. Jeśli kasza kuskus można spróbować dać połowę dyni, nie jestem pewny czy oddaje smak. Jaglaną lepiej bez, dynię podać osobno.}

\commentskip

\recipend

\section{Mielone tofu erVegan}

\begin{ingreds}
\item{naturalne tofu, 300g}
\item{płatki drożdżowe, 0.25 szklanki}
\item{sos sojowy, 3--4 łyżki}
\item{papryka wędzona, 1.5 łyżeczki}
\item{czosnek w proszku, 1 łyżeczka}
\item{cebula w proszku, 1 łyżeczka}
\end{ingreds}

\recipara{Tofu odsącz, w większej misce rozgnieć widelcem na mniejsze kawałki. Piekarnik nagrzej do 200 stopni.}
\recipara{W małej szklance wymieszaj wszystkie przyprawy, dodaj do rozkruszonego tofu. Dodaj sos sojowy i dokładnie wymieszaj. Dopraw pieprzem.}
\recipara{Podsmaż tofu na mocno rozgrzanej patelni z odrobiną oleju, nie mieszaj, pozwól się lekko przypalić. Wyłóż papier do pieczenia na blasze i równomiernie rozłóż tofu. Po włożeniu zmniejsz temperaturę do 150 stopni i piecz przez $\sim$10min, zerkając czy się nie przypala. Mieszaj co kilka minut i powtarzaj aż do żądanej konsystencji, około 20min.}

\commentskip

\comment{Sprawdź wykonane bardzo dokładnie z wszystkimi przyprawami. Ma potencjał.}

\comment{Nie przesadź w piekarniku, żeby nie zeschło się za bardzo.}

\comment{Można użyć do farszu.}

\recipend

\section{Tofu zmieniające życie}

\begin{ingreds}
\item{tofu, dwie kostki, 360g}
\item{mąka ziemniaczana lub kukurydziana, 3 łyżki}
\item{miód, syrop klonowy, inny słodzik, 3 łyżki}
\item{sos sojowy, 3 łyżki}
\item{imbir, 2--3cm}
\item{woda, 0.25 szklanki}
\item{szczypiorek}
\opitem{czosnek, 1--2 ząbki}
\opitem{wędzona papryka, 0.5 łyżeczki}
\opitem{ocet ryżowy/z białego wina/sok z limonki, 2 łyżki}
\opitem{olej sezamowy, 2 łyżki}
\opitem{sezam}
\opitem{kolendra}
\end{ingreds}

\recipara{W pierwszej wersji, z czosnkiem, tofu pokroić, wsypać mąkę ziemniaczaną do woreczka i potrząsać. Na dnie dużej patelni rozgrzać tyle oleju, żeby pokrywał całe dno. Smażyć przez 3--4min z każdej strony, aż do ciepłęgo odcienia panierki. W drugiej wstawic piekarnik na 220 stopni, dobrze osuszyć, obtoczyć w mące, wędzonej papryce i pieprzu, polać 3 łyżkami oleju, znowu obtoczyć. Piec $\sim$20min, do zezłocenia, potrząsając blachą w połowie.}

\recipara{Do smażonego tofu przygotować talerz z ręcznikiem papierowym i sos. Drobno zetrzeć imbir na $\sim$2 łyżeczki oraz czosnek. Dodać do szklanki z sosem sojowym, słodzikiem i wodą, wymiesząć. Do pieczonego tofu wlej na patelnię sos sojowy, słodzik, ocet ryżowy, olej sezamowy, wodę i imbir. Zagotuj i gotuj przez $\sim$5min do zgęstnienia.}

\recipara{Usmażone tofu odłożyć na chwilę na papierowy ręcznik. Na mniejszą patelnię wlać sos i podgrzewać aż zacznie intensywnie bulgotać, dodać tofu. Mieszeć cały czas, aż oblepi się sosem. Upieczone tofu przełożyć na patelnię, obtoczyć w sosie potrząsając ją. Gotować minutę lub dwie na małym ogniu. Posypać sezamem i kolendrą.}

\commentskip

\comment{Jako przekąska można podać pieczone tofu bez sosu. Dodać wtedy o pół łyżeczki wędzonej papryki więcej i dodatkowe 0.5 łyżeczki soli.}

\recipend

\section{Bibim Guksu, pikantny makaron na zimno}

\begin{ingreds}
\item{makaron, porcja dla 2--3 osób}
\item{ogórek}
\item{jajko}
\item{kimchi}
\item{wodorosty nori}
\item{cebula dymka}
\item{marchewka, kapusta lub inne warzywa}
\item{gochujang, pasta chilli, 2--3 łyżki}
\item{gochugar, płatki chilli, 1 łyżka}
\item{ocet ryżowy lub inny biały, 2 łyżki}
\item{cukier, 2 łyżki}
\item{sos sojowy, 1 łyżka}
\item{olej sezamowy, 1 łyżka}
\item{czosnek, 1 ząbek}
\item{sezam}
\end{ingreds}

\recipara{Makaron i jajko ugotować. Makaron odcedzić i wypłukać w zimnej wodzie. Posiekać czosnek i wraz z wszystkimi przyprawami wymieszać w miseczce (gochugar, gochujang, sos sojowy, olej sezamowy, ocet, cukier, 1--2 łyżki wody).}
\recipara{Ogórka, kimchi, szczypiorek, marchewkę drobno pokroić. Nori uprażyć nad palnikiem lub na suchej patelni. Pokruszyć lub pociąć nożyczkami.}
\recipara{Wszystkie składniki ułożyć w okręgu na makaronie, ugotowane jajko przekroić i ułożyć na środku. Polać sosem i posypać sezamem.}

\commentskip

\recipend
